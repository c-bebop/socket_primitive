\documentclass[xcolor=dvipsnames]{article}

% Bibtex
\usepackage{cite, authordate1-4}
\usepackage{url}

\usepackage[pdftex]{graphicx}

% Open Sans Package!
\usepackage[default,scale=0.95]{opensans}
\usepackage[T1]{fontenc}

\usepackage{transparent}

\usepackage[dvipsnames]{xcolor}
\definecolor{blue}{cmyk}{0.81,0.26,0,0.48}
\definecolor{code_backgrond}{cmyk}{0.02,0.01,0,0.07}
\definecolor{red}{cmyk}{0,0.76,0.76,0.48}
\definecolor{lbcolor}{rgb}{1,1,1}
\definecolor{mygray}{rgb}{0.3,0.3,0.3}

% GLOSSAR BEGIN
\usepackage[acronym]{glossaries}

\usepackage{tikz}

% Listing for Code
\usepackage{listings}


\lstset{ %
	language=Erlang,                % choose the language of the code
	basicstyle=\ttfamily\footnotesize\color{black},       % the size of the fonts that are used for the code
	numbers=left,                   % where to put the line-numbers
	numberstyle=\color{black}\footnotesize,      % the size of the fonts that are used for the line-numbers
	stepnumber=1,                   % the step between two line-numbers. If it is 1 each line will be numbered
	numbersep=3pt,                  % how far the line-numbers are from the code
	showspaces=false,               % show spaces adding particular underscores
	showstringspaces=false,         % underline spaces within strings
	showtabs=false,                 % show tabs within strings adding particular underscores
	%frame=single,           % adds a frame around the code
	tabsize=2,          % sets default tabsize to 2 spaces
	captionpos=b,           % sets the caption-position to bottom
	breaklines=true,        % sets automatic line breaking
	breakatwhitespace=false,    % sets if automatic breaks should only happen at whitespace
	escapeinside={\%*}{*)},         % if you want to add a comment within your code
	keywordstyle=\color{red},
	commentstyle=\color{mygray},
	stringstyle=\color{blue},
	backgroundcolor=\color{code_backgrond}
}

% Generate the glossary
\makenoidxglossaries

%Term definitions
\newacronym{os}{OS}{Operating System}
\newacronym{mpi}{MPI}{Message-Passing Interface}
\newacronym{rpc}{RPC}{Remote Procedure Calls}
\newacronym{tcp}{TCP/IP}{Transmission Control Protocol / Internet Protocol}
\newacronym{xml}{XML}{Extensible Markup Language}
\newacronym{json}{JSON}{JavaScript Object Notation}
\newacronym{ipc}{IPC}{Inter-Process Communication}
\newacronym{rtp}{RTP}{Real-Time Transport Protocol}
\newacronym{udp}{UDP}{User Datagram Protocol}
\newacronym{osi}{OSI}{Open Systems Interconnection Reference Model}
\newacronym{iso}{ISO}{International Organization for Standardization}
\newacronym{fifo}{FIFO}{First In - First Out}
\newacronym{pid}{PID}{Process Identifier}
\newacronym{mpif}{MPIF}{Message-Passing Interface Forum}
\newacronym{hpc}{HPC}{High-Performance Computing}
\newacronym{ascii}{ASCII}{American Standard Code for Information Interchange}
\newacronym{oop}{OOP}{Object-oriented programming}

% GLOSSAR END

% For HTML conversion
%\usepackage{tex4ht}

\title{
\scshape{
\begin{large}
Definition of the term\\
\end{large} 
\textbf{\textcolor{blue}{Socket Primitive}}\\
\begin{large}
In Distributed Systems
\end{large}
}}

\author{Florian Willich \\ Hochschule f\"ur Technik und Wirtschaft Berlin \\ University of Applied Sciences Berlin \\ Course: Distributed Systems \\ Lecturer: Prof. Dr. Christin Schmidt}
        
\date{\today}

\begin{document}

\pagenumbering{gobble} 

\begin{titlepage}

\maketitle

\begin{abstract}
Socket Primitive is a heavily used term associated with network operations in computer science. A socket is a communication end point of a computer system. The word \textit{primitive} is often used to name an instruction which represents a self-contained unit on the current abstraction level. Therefore, a \textit{Socket Primitive} is a socket related function that is provided by various high-level programming language libraries.
\end{abstract}

\end{titlepage}

\newpage

\tableofcontents

\newpage

\pagenumbering{arabic}

% Socket Primitive in anfuehrungszeichen

\section{\scshape{\textcolor{blue}{Introduction}}} \label{introduction}

During my work on a technical report\footnote{Willich, Florian. 2015. Introductory Guide to Message Passing in Distributed Computer Systems. \url{https://github.com/c-bebop/message_passing/blob/master/technical_report/message-passing_distributed-systems_florian-willich.pdf}} I came across the term \textit{Socket Primitive} which was a heavily used term when describing operations with the \gls{tcp}, which in one instance was used as follows \cite[p. 141, ch. 4.3.1]{tanenbaum}:

\begin{quote}\label{tanenbaum_quote}
[...] A socket forms an abstraction over the actual communication end
point that is used by the local operating system for a specific
transport protocol. In the following text, we concentrate on the socket
primitives for TCP, which are shown in Fig. 4-14. [...] 
\end{quote}

\noindent Inevitably the question came up: What exactly does the term \textit{Socket Primitive} define?  With this technical report, I will first clarify what a \textit{Socket} is and what \textit{Primitive} means in such a context. Furthermore, I will describe the term \textit{Socket Primitive} by using examples and ultimately define the term.\\

\noindent I also would like to give special thanks to Maarten van Steen, co-author of the quoted book \cite{tanenbaum}, who aided me with some clarification at the beginning of my research.\\

\noindent The reader is asked to keep in mind that although I am writing in a more abstract and general way about \textit{Socket Primitives}, it is always considered to operate with a UNIX \gls{os}.

\section{\scshape{\textcolor{blue}{Definition of the terms Socket and Primitive}}}

\subsection{\scshape{\textcolor{blue}{Socket}}}\label{introduction_socket}

A socket is a communication end point of a computer system. Whenever two computer systems shall communicate with each other, a connection between those systems has to be established by using interconnected sockets \cite[p. 553, ch. 6.5.2]{computer_networks}.\\

\noindent A socket is always considered to be used associated with a protocol that implements a model of communication. \gls{ipc} will only be successful when both processes use the same protocol. Common protocols are the \gls{udp} or \gls{tcp} for example. The used protocol determines which low-level mechanisms are used to transmit and receive data \cite[p. 427 - 427, ch. 16]{GNU_C_library_manual}.

\subsection{\scshape{\textcolor{blue}{Primitive}}}

\subsubsection{\scshape{\textcolor{blue}{Meaning in Computer Science}}}

In computer science, the word \textit{primitive} is often used to name an instruction which represents a self-contained unit on the current abstraction level. This also means that there is usually no need for further description of what this instruction is composed of, except for descriptions of the semantics and logic the \textit{primitive} introduces.\\

\noindent For example: The GNU C Reference Manual uses the term \textit{Primitive Data Types} to describe the built-in data types in the programming language C \cite[p. 8, ch. 2]{GNU_C_manual}. The following table shows a selection of primitive data types of a C language, their characteristics and the memory allocation on 64-bit systems:\\

\begin{center}
\begin{tabular}{ | l | l | l | }
\hline
\textcolor{blue}{\textbf{Primitive}} & \textcolor{blue}{Stores} & \textcolor{blue}{Memory Allocation}\\
\hline
\textbf{char} & \glsdisp{ascii}{ASCII} characters & 1 byte\\
\textbf{int} & integers  & 4 byte\\
\textbf{float} & floating point real numbers & 4 byte\\
\textbf{double} & double precision & 8 byte\\
& floating point real numbers &\\
\hline
\end{tabular}
\end{center}

\noindent As long as a programmer knows what each built-in \textit{Primitive Data Type} represents and how to use it, he must not necessarily know what assembly code the compiler is generating to write code in a high level language. On the contrary, a compiler builder has to.\\

\noindent Please see Appendix \ref{c++_data_types} for the corresponding implementation of the application which prints out the size allocated by primitive data types in C++.

\subsubsection{\scshape{\textcolor{blue}{Meaning in Linguistic Science}}}

\noindent The \textit{Oxford Advanced Learners Dictionary} has no definition of the word \textit{primitive} that fits for the use in computer science \cite[p. 1197]{oxford_dictionary}. The online version \url{http://www.oxfordlearnersdictionaries.com} offers\footnote{Original URL: \url{http://www.oxfordlearnersdictionaries.com/definition/english/primitive_1?q=primitive}} merely the basic etymology of primitive:

\begin{quote}
late Middle English (in the sense 'original, not derivative'): from Old French primitif, -ive, from Latin primitivus 'first of its kind', from primus 'first'.
\end{quote}

\noindent However, also this additional definition does not correspond to its use in computer science, which indicates that the term is commonly used in a 'sloppy' way. 

\section{\scshape{\textcolor{blue}{TCP/IP Socket Primitives}}} \label{socket_primitives}

As already mentioned above (chapter \ref{introduction_socket}), a socket always communicates in association with a protocol. For the programmer in a high-level programming language, protocols specify the interface for operation with sockets.\\

\noindent The following table lists the most common \gls{tcp} \textit{Socket Primitives} with a short description (\cite{IBM_Anupama} and \cite[p. 142, ch. 4.3.1]{tanenbaum}):\\

\begin{center}
\begin{tabular}{ | l | l | } 
\hline
\textcolor{blue}{\textbf{Primitive}} & \textcolor{blue}{Service description}\\
\hline
\textbf{Socket} & creates a new socket e.g. communication end point\\
\textbf{Bind} & associates a local address with a socket\\
\textbf{Listen} & allows to accept new incoming connections to a socket\\
\textbf{Accept} & blocks until a connection request\\
\textbf{Connect} & connects to a socket\\
\textbf{Send} & sends a message to a socket\\
\textbf{Receive} & reads a message from a socket\\
\textbf{Close} & aborts a connection\\
\hline
\end{tabular}
\end{center}
\label{tab:tcp_primitives}

\noindent The following definition of the function \textit{send()} is the implementation of the \textit{Socket Primitive} \textbf{Send} by the GNU C Library \cite[p. 457, ch. 16.9.5.1]{GNU_C_library_manual}:

\begin{lstlisting}[language=C, numbers=none]
ssize_t send(  	int socket, 
								const void *buffer, 
								size_t size, 
								int flags)
\end{lstlisting}

\noindent A short description of the parameters:

\begin{itemize}
\item socket: The socket file descriptor.
\item buffer: Memory address to the data which shall be sent.
\item size: The size in bytes of the data that shall be sent.
\item flags: Additional information for the sending instruction \cite[p. 459, ch. 16.9.5.3]{GNU_C_library_manual}.
\end{itemize}

\noindent The function either returns the number of bytes which have been transmitted or $-1$ if an error occurred. For more detailed information reading the referenced GNU C Library Reference Manual \cite{GNU_C_library_manual} on page 457 in chapter 16.9.5.1 is recommended.\\

\noindent The \textit{send()} function is called by the application in user mode which will then call a number of other functions in kernel mode. These then operate on the socket layer, protocol layer and interface layer \cite[p. 2, 16]{IBM_Anupama}. This implicitly means that even if the interface for sending a message to a socket is relatively easy to use, the actual call stack and possible errors introduce another level of complexity. However, as long as the programmer does not want to treat every possible error he does not need to understand how the \textit{send()} function actually performs the desired instruction but instead is able to completely rely on its provided functionality.\\

\noindent It is also important to mention that this was an example of how the \textit{Send} primitive could be defined as a function. Other C libraries can differ in their definition and provided functionality, not to mention other programming languages.

\section{\scshape{\textcolor{blue}{Definition}}} \label{definition}

Now that the term \textit{Socket Primitive} is more clarified, the following definition of primitives by Andrew S. Tanenbaum gives a very good summary \cite[p. 38, ch. 1.3.4]{computer_networks}:

\begin{quote}
A service is formally specified by a set of primitives (operations) available to user processes to access the service. These primitives tell the service to perform some action or report on an action taken by a peer entity. If the protocol stack is located in the operating system, as it often is, the primitives are normally system calls. These calls cause a trap to kernel mode, which then turns control of the machine over to the operating system to send the necessary packets.
\end{quote}

\noindent The following therefore is the working definition of a \textit{Socket Primitive}:

\begin{quote}

\textit{A \textit{Socket Primitive} is a socket related function that is provided by various high-level programming language libraries. A socket is a communication endpoint of a UNIX \gls{os}. Socket Primitives are called primitive because they represent self-contained low-level units that operate with sockets.}

\end{quote}

\newpage

\begin{appendix}

\section{Primitive Data Types in C++}\label{c++_data_types}

\lstinputlisting[language=C++]{../dataype_primitive_c++_application/main.cpp}

\end{appendix}

\newpage

\printnoidxglossaries

\bibliographystyle{authordate1}
\bibliography{lib}

\vfill
\begin{center}
This document was written with \LaTeX 
\\Typeface: Open Sans by Steve Matteson.
\end{center}

% To compile bibtex and latex manual:
% bibtex belegarbeit.aux
% latex belegarbeit.tex

\end{document}