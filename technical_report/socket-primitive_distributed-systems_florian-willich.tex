\documentclass[xcolor=dvipsnames]{article}

% Bibtex
\usepackage{cite, authordate1-4}
\usepackage{url}

\usepackage[pdftex]{graphicx}

% Open Sans Package!
\usepackage[default,scale=0.95]{opensans}
\usepackage[T1]{fontenc}

\usepackage{transparent}

\usepackage[dvipsnames]{xcolor}
\definecolor{blue}{cmyk}{0.81,0.26,0,0.48}
\definecolor{code_backgrond}{cmyk}{0.02,0.01,0,0.07}
\definecolor{red}{cmyk}{0,0.76,0.76,0.48}
\definecolor{lbcolor}{rgb}{1,1,1}
\definecolor{mygray}{rgb}{0.3,0.3,0.3}

% GLOSSAR BEGIN
\usepackage[acronym]{glossaries}

\usepackage{tikz}

% Listing for Code
\usepackage{listings}


\lstset{ %
	language=Erlang,                % choose the language of the code
	basicstyle=\ttfamily\footnotesize\color{black},       % the size of the fonts that are used for the code
	numbers=left,                   % where to put the line-numbers
	numberstyle=\color{black}\footnotesize,      % the size of the fonts that are used for the line-numbers
	stepnumber=1,                   % the step between two line-numbers. If it is 1 each line will be numbered
	numbersep=3pt,                  % how far the line-numbers are from the code
	showspaces=false,               % show spaces adding particular underscores
	showstringspaces=false,         % underline spaces within strings
	showtabs=false,                 % show tabs within strings adding particular underscores
	%frame=single,           % adds a frame around the code
	tabsize=2,          % sets default tabsize to 2 spaces
	captionpos=b,           % sets the caption-position to bottom
	breaklines=true,        % sets automatic line breaking
	breakatwhitespace=false,    % sets if automatic breaks should only happen at whitespace
	escapeinside={\%*}{*)},         % if you want to add a comment within your code
	keywordstyle=\color{red},
	commentstyle=\color{mygray},
	stringstyle=\color{black},
	backgroundcolor=\color{code_backgrond}
}

% Generate the glossary
\makenoidxglossaries

%Term definitions
\newacronym{os}{OS}{Operating System}
\newacronym{mpi}{MPI}{Message-Passing Interface}
\newacronym{rpc}{RPC}{Remote Procedure Calls}
\newacronym{tcp}{TCP/IP}{Transmission Control Protocol / Internet Protocol}
\newacronym{xml}{XML}{Extensible Markup Language}
\newacronym{json}{JSON}{JavaScript Object Notation}
\newacronym{ipc}{IPC}{Inter-Process Communication}
\newacronym{rtp}{RTP}{Real-Time Transport Protocol}
\newacronym{udp}{UDP}{User Datagram Protocol}
\newacronym{osi}{OSI}{Open Systems Interconnection Reference Model}
\newacronym{iso}{ISO}{International Organization for Standardization}
\newacronym{fifo}{FIFO}{First In - First Out}
\newacronym{pid}{PID}{Process Identifier}
\newacronym{mpif}{MPIF}{Message-Passing Interface Forum}
\newacronym{hpc}{HPC}{High-Performance Computing}
\newacronym{ascii}{ASCII}{American Standard Code for Information Interchange}

% GLOSSAR END

% For HTML conversion
%\usepackage{tex4ht}

\title{\scshape{
\begin{small}
Definition of the term\\
\end{small} 
\textbf{\textcolor{blue}{Socket Primitive}}\\
\begin{small}
In Distributed Systems
\end{small}
}}

\author{Florian Willich \\ Hochschule f\"ur Technik und Wirtschaft Berlin \\ University of Applied Sciences Berlin \\ Course: Distributed Systems \\ Lecturer: Prof. Dr. Christin Schmidt}
        
\date{\today}

\begin{document}

\pagenumbering{gobble} 

\begin{titlepage}

\maketitle

\begin{abstract}
Abstract...
\end{abstract}

\end{titlepage}

\newpage

\tableofcontents

\newpage

\pagenumbering{arabic} 

\section{\scshape{\textcolor{blue}{Introduction}}} \label{introduction}

During my work on a technical report I came across the term \textit{Socket Primitive} which was a heavily used term when describing operations with the \gls{tcp} \cite[p. 141, ch. 4.3.1]{tanenbaum}:

\begin{quote}
[...] A socket forms an abstraction over the actual communication end
point that is used by the local operating system for a specific
transport protocol. In the following text, we concentrate on the socket
primitives for TCP, which are shown in Fig. 4-14. [...] 
\end{quote}

\noindent Inevitably the question came up what exactly the term \textit{Socket Primitive} defines?\\

\noindent It appears to be that either the term \textit{Socket Primitive} is used when\\

\noindent Please keep in mind that although I am writing in a more abstract and general way about \textit{Socket Primitives}, it is always considered to operate with a UNIX \gls{os}.

\subsection{\scshape{\textcolor{blue}{Socket}}}

A Socket is a communication end point of a computer system. Whenever two computer systems shall communicate with each other, a connection between those has to be established and therefore a connection between the local sockets has to be established \cite[p. 553, ch. 6.5.2]{computer_networks}.

\subsection{\scshape{\textcolor{blue}{Primitive}}}

In computer science, the word \textit{primitive} is often used to name an instruction which represents a self-contained unit. Whenever the word \textit{primitive} is used, the most basic instruction on the current abstraction level is ment. This also means that there will probably be no further description of what this instruction is composed of.\\

\noindent For example: The GNU C Reference Manual uses the term \textit{Primitive Data Types} to describe the built-in data types in the programming language C \cite[p. 8, ch. 2]{GNU_C_manual}. The following table shows a selection of primitive data types of a C language, their characteristics and the memory allocation on 64-bit systems. Please see Appendix \ref{c++_data_types} for implementation details:\\

\begin{center}
\begin{tabular}{ | l | l | l | }
\hline
\textcolor{blue}{Primitive} & \textcolor{blue}{Characteristic} & \textcolor{blue}{Memory Allocation}\\
\hline
char & stores \glsdisp{ascii}{ASCII} characters & 1 byte\\
int & stores integers  & 4 byte\\
float & stores floating point real numbers & 4 byte\\
double & stores double precision & 8 byte\\
& floating point real numbers &\\
\hline
\end{tabular}
\end{center}

\noindent As long as a programmer knows what each built-in \textit{Primitive Data Type} represents and how to use it, he must not necessarily know what assembly code the compiler is generating to write code in a high level language. On the contrary, a compiler builder has to.\\

\noindent The \textit{Oxford Advanced Learners Dictionary} has no definition of the word \textit{primitive} that fits for the use in computer science \cite[p. 1197]{oxford_dictionary}. Nevertheless, the online version \url{http://www.oxfordlearnersdictionaries.com} offers an additional word origin:

\begin{quote}
late Middle English (in the sense 'original, not derivative'): from Old French primitif, -ive, from Latin primitivus 'first of its kind', from primus 'first'.
\end{quote}

\noindent \small{Source URL: \url{http://www.oxfordlearnersdictionaries.com/definition/english/primitive_1?q=primitive}}\\

\noindent 

\section{\scshape{\textcolor{blue}{TCP/IP Socket Primitives}}} \label{socket_primitives}

The following table lists the most common \gls{tcp} socket primitives with a short description \cite{IBM_Anupama}:\\

\begin{center}
\begin{tabular}{ | l | l | } 
\hline
\textcolor{blue}{Primitive} & \textcolor{blue}{Service description}\\
\hline
Socket & creates a new socket e.g. communication end point\\
Bind & associates a local address with a socket\\
Listen & allows to accept new incoming connections to a socket\\
Accept & blocks until a connection request\\
Connect & connects to a socket\\
Send & sends a message to a socket\\
Receive & reads a message from a socket\\
Close & aborts a connection\\
\hline
\end{tabular}
\end{center}
\label{tab:tcp_primitives}

Figure \ref{tab:tcp_primitives} describes the \gls{tcp} socket primitives \cite[p. 142, ch. 4.3.1]{tanenbaum}

\section{\scshape{\textcolor{blue}{Definition}}} \label{definition}

\begin{quote}

\textit{A socket primitive is a socket related function that is provided by various standard libraries for various high-level programming languages. It provides an interface for calling operating system specific instructions. Therefore, a 'socket primitive' is called primitive because it is the lowest level interface for a programmer who is writing in a high level programming language to operate with network protocols.}

\end{quote}

\begin{appendix}

\section{Primitive Data Types in C++}\label{c++_data_types}

\lstinputlisting[language=C++]{../dataype_primitive_c++_application/main.cpp}

\end{appendix}

\printnoidxglossaries

\newpage

\bibliographystyle{authordate1}
\bibliography{lib}

\vfill
\begin{center}
This document was written with \LaTeX 
\\Typeface: Open Sans by Steve Matteson.
\end{center}

% To compile bibtex and latex manual:
% bibtex belegarbeit.aux
% latex belegarbeit.tex

\end{document}